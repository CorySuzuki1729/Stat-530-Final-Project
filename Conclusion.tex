
% Conclusion File

\chapter{Concluding Remarks}
\thispagestyle{empty}
\section{Summary}

In this experiment, we investigated the physical performance of hikers affected by two blocking variables, in which temperature and humidity levels were chosen. The Analysis of Variance concludes that temperature plays the most significant role in hiking. 

In addition, the post-ANOVA analysis supported this conclusion further by showing that for temperature, temperatures 70 degrees Fahrenheit and 75 degrees Fahrenheit are statistically significant. For humidity levels in percentages, humidity levels $5\%$, $10\%$, $15\%$ are statistically significant. These conclusions mean that higher temperatures are detrimental to a given hiker's health and physical endurance. Therefore, a good recommendation to solve this issue and to decrease the negative effects of high temperatures is to advocate widespread hydration drinks or the production of specialized shirts with cold packs that can cool hikers.

If future work is done on the replicated Latin Square (Case I) design with the field data collected, one potential improvement that can be made to improve the robustness of the analysis is to provide more replications of the experiment and consider higher levels of both temperature and humidity. For example, further analysis can be done with $p>4$ if more data is collected, and our conclusion may change due to the variation in the distances traversed since different hikers attain different physical activity levels. This would be a more interesting experiment, although a more laborious one that may yield intriguing results. We have the hopeful intention that this experiment will provide some insight on how to improve the safety of prospective adventurers and hikers.