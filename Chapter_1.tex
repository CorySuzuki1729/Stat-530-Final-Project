
% Chapter 2 File

\chapter{Introduction}
\label{chapter1}
\thispagestyle{empty}

\section{Motivation for Research: Why is the Performance of Hikers Important?}
Hiking is an excellent recreational activity that any individual can enjoy regardless of their hiking skill level. It is often nice to experience the outdoors from a new perspective and to see the local wildlife as one climbs rocky mountains covered in boulders or traverses through the extreme weather of desert landscapes. However, hiking does come with certain risks and there are numerous factors that can affect an individual's hike. For example, the weather can be too hot or too cold, and that can affect respiratory function which is required for long-distance hikes. The altitude of a mountain can also wear out hikers who are less experienced.
\\\\
More than often, hikers need to check the humidity percentage as it is a factor that accelerates your body's dehydration. If the trail is located in the mountains or near coastlines, the humidity level will be higher than the humidity in typical urban and suburban areas. Many hiking safety guides such as the site \emph{Hiking with Shawn}, warn potential hikers of dehydration due to the increased amount of bodily perspiration as a result of both high temperatures and high humidity levels \cite{key1}. The safety of both experienced and inexperienced hikers is important and above all, a guaranteed way to provide the adventure and excitement of experiencing the great outdoors, which proposes the question: Can we construct and conduct an experiment to analyze and draw meaningful statistical conclusions to facilitate in the development of better gear and technology for hikers? The analysis discussed in further detail in this paper provides a meaningful answer to this question.